\documentclass{article}
\usepackage{listings}
\usepackage{color}
\title{Lab 3: A Machine for Sorting Bricks}
\author{Michael Carpenter \\ Benjamin Liebersohn}
\date{\today}
\begin{document}
\maketitle

\section{Introduction}
For this lab, we were tasked to build a Lego NXT robot capable of sorting colored bricks into one of two categories: light or dark.
Furthermore, it had to be able to do this for a random sequence of bricks of varying color.
The main value of this was to build an autonomous robot through the integration of hardware and software, that could perform a specific task accurately and reliably. 
This had particular value given it reduced the amount of new concepts that had to be wrangled in order to obtain a robot that could actually perform the given task at least passably well.
Only one sensor and one motor was needed to perform the actual sorting and the code involved to control the robot was sort and simple.
That left room for what was probably the most intensive part of the lab: testing the complete robot itself. It took many iterations in order to arrive at our final design.

\section{Hardware}
Our robot is is designed to organize blocks based on color into two groups: light (white, yellow), and dark (black, blue). 
The way our sorter determines whether a block is light or dark is with a light sensor. 
By shining a light on the object, the light sensor can measure the reflectivity of a given surface. 
A less reflective surface is darker to the light sensor. 
In our robot, a red LED illuminates a brick, light which is then reflected back to the light sensor. 
Our robot begins its sorting by taking hundreds of light readings and aggregating them to find the ambient light reading. 
Once the robot has completed this, the algorithm begins to take light samples in order to determine the state the motor is in. 
If a brick is placed in front of the sensor, the light sensor samples will change, and the algorithm will calculate whether the object in front is relatively light or dark. 
Once that has been determined, the motor powers a lever which pushes the brick out of the sorter in one of two directions.
Either the lever is turned counter clockwise to push a light brick to the left or counter clockwise to push the dark brick to the right. 
From there, a bumper and deflector guide the brick out of the sorter, into the cups below. 
Occasionally, bricks may jam in front of the light sensor, which makes the lever push until the gears which provide the lever with power begin to slip, which brings bricks away from the edge.

\section{Software}
For the software component, we used a well documented NXT library written in the Haskell programming language, as it was the quickest way to get a program written that worked.
The implementation consists of two function, one to calibrate the robot and the other to actually sort the brick.
When the robot first fires up, it enters into the calibration function which takes a series of readings with the light sensor and then finds the largest reading.
This reading is used as a threshold reading for the sorting function.
The sorting function starts off by taking a reading with the light sensor.
It also takes the threshold value calculated previously during the calibration stage and uses it as a sensitivity baseline on which to decide whether the incoming sensor reading is of a Lego brick or not.
If the reading drops below the calculated threshold, then the reading is interpreted as the fact that there is no brick currently in front of the sensor to be sorted and that the sorter function should simply return.
If the reading is above the threshold, then the reading is compared against a pivot.
Depending on the boolean value of this comparison, the robot will either rotate its single motor clockwise or counter-clockwise.
This caused the robots arm to swipe the Lego left or right.
This function is run in an infinite loop until the program is manually stopped by the user.

\section{Tests}[!ht]
\begin{table}
  \begin{center}
    \begin{tabular}{| c | c | c | c | c | c | c | c | c | c | c |}
      \hline 
      Color   & T1 & T2 & T3 & T4 & T5 & T6 & T7 & T8 & T9 & T10 \\
      \hline
      Black   &   D &  D  &  D  &  D  &  D  &  D  &  D  &  D  &  D  &  D   \\
      Blue    & D   &  D  &  D  &  D  &  D  &  D  &  D  &  D  &  D  &  D   \\
      Green   &  D  &  D  &  D  &  D  &  D  &  D  & D   &  D  & D   &  D   \\
 Light Green  &  D  & D   &  D  &  D  &  D  &  D  &  D  &  D  & D   &  D   \\
      Red     &  L  & L   &   L &  L  &  L  &  L  &  L  &  L  & L   &  L   \\
      Yellow  & L   &  L  & L   & L   &  L  & L   & L   &  L  & L   &  L   \\
      Pink    & L   &  L  & L   & L   & L   &  L  &  L  & L   & L   &  L   \\
      White   &  L  &  L  & L   & L   &  L  & L   &  L  &  L  &  L  &  L   \\
      \hline
    \end{tabular}
  \end{center}
  \caption{A table with 10 trials for each color brick.}
\end{table}

\section{Discussion}
In a future design, we would like to use slightly different methods to get more useful reflectivity readings. 
Because the LED we used emits a red spectrum light, some colors of Lego bricks reflect light back in ways which are not how the bricks are seen by us. 
This was an issue when we wanted to sort the light green Lego brick, which reflected red spectrum light poorly, and thus gets incorrectly sorted as dark.
Instead, using the white incandescent light supplied with the kit may provided more balanced spectrum to reflect back to the light sensor, which would give us readings more similar to how the bricks look under natural light. 
Another change to make to the methods involving the light sensor would be to continue to sample ambient readings throughout the sorting process. 
Instead of relying on only an initial sampling, additional samples of ambient light could be taken when bricks are not in front of the sensor. 
The sorter could know if a brick is being examined by using the ultrasonic sensors to determine if something is directly in front of the light sensor. 
This way, additional samples of ambient light are taken, allowing the ambient light reading to continuously change.
Another aspect of our design which we could improve further would be our accuracy at landing bricks in the cup. 
While most bricks do land in the correct cup, sometimes a brick might bounce back out, or get pushed back towards the motor. 
The best solution to these problems will likely involve looking for ways to lower the sorting platform and arm, in order to lessen the likelihood of failure due to the bricks bouncing or falling inaccurately.  

\newpage
\appendix
\section{Code}
\definecolor{mygreen}{rgb}{0,0.6,0}
\definecolor{mygray}{rgb}{0.5,0.5,0.5}
\definecolor{mymauve}{rgb}{0.58,0,0.82}
\lstset{
  backgroundcolor=\color{white},
  basicstyle=\footnotesize,
  breakatwhitespace=false,
  breaklines=true,
  captionpos=b,
  commentstyle=\color{mygreen},
  deletekeywords={...},
  extendedchars=true,
  frame=single,
  keepspaces=true,
  keywordstyle=\color{blue},
  language=Haskell,
  otherkeywords={*,...},
  numbers=left,
  numbersep=5pt,
  numberstyle=\tiny\color{mygray},
  rulecolor=\color{black},
  showspaces=false,
  showstringspaces=false,
  showtabs=false,
  stepnumber=2,
  stringstyle=\color{mymauve},
  tabsize=2,
  title=\lstname
}

/lstinputlisting[language=Haskell]{sorter.hs}
\end{document}
